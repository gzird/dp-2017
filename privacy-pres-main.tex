\begin{frame}{Cool things we learned in class}{including Chernoff bounds}
  \begin{itemize}
    \item Laplace mechanism: $|Q| \approx n^2$,\quad $|X| = n^{\omega(1)}$ 
    \item Noisy histogram: $|X| \approx n^2$,\quad $|Q| = n^{\omega(1)}$
    \item PMW: if $|X|,|Q| \in \poly(n)$
  \end{itemize}
  \vspace{1em}
  We require to be \textbf{computationally} efficient
  \\$\rightarrow$ especially while giving this talk
\end{frame}

% \begin{frame}{Outline}{Content of this Presentation}
% 	\begin{enumerate} [<+->]
% 		\item item 1
% 		\item item 2
% 	\end{enumerate}
% \end{frame}

\begin{frame}{Memories...}{A flashback from the algo class - fa'15}
  Solving linear programs
  
  \begin{itemize}
  \item Dantzig'47: Simplex method 
  \item Khachiyan '79: Ellipsoid method ($n^4L$, n vars, L bits)
  \item ...
  \end{itemize}

  But...
  \begin{itemize}
    \item Simplex is not worst case polynomial time
    \item Klee-Minty polytope requires exp time
  \end{itemize}
  \vspace{1em}
  Can we ``cook'' hard instances for privacy?
	\begin{itemize}
		\item Yes, if you believe in crypto!
		\item If not, let us know ASAP of your results.
	\end{itemize}
\end{frame}

\begin{frame}{The Model}
  \begin{itemize}
    \item Non-interactive setting
    \item Fix beforehand a query set $Q$
    \item Curator sanitizes/privatizes a dataset $X^n$ for all $q \in Q$
      \begin{itemize}
      \item Synthetic dataset - Can run my queries, it's a dataset!
      \item Arbitrary output - Need an evaluator
      \end{itemize}

    \item Throw away the original data and the curator~\includegraphics[width=0.8cm]{trollface}
  \end{itemize}
\end{frame}

\begin{frame}{Hardness for synthetic datasets~\cite{conf/stoc/DworkNRRV09}}
  (super-strong) Signatures from OWF~\cite{conf/stoc/NaorY89,conf/stoc/Rompel90}
  \begin{itemize}
    \item Data universe $X$: pairs (msg, sig) = \fbox{$(m,\sigma)$}
    \item Query set $Q$: \fbox{$q_{\vk}(m,\sigma) = 1$ iff $\Verify_{vk}(m,\sigma) =
      1$}
    \item Items of $Q,X$ are in $\zo^\secp \Longrightarrow$  both $|Q|$, $|X|$
      have size $2^\secp$.
    \item Sample a dataset $X^n$ with the \textbf{same} secret-key:
      \begin{itemize}
        \item $m \sample \zo^\secp$ 
        \item $\sigma \sample \Sign(\sk, m)$
        \item Output $(m, \sigma)$ 
       \end{itemize}
    \end{itemize}
    
    \vspace{2em}
    \textbf{Goal:} Release a dataset that preserves some fractional count of
    the sigs (depending on accuracy) that verify for a key-pair $(\sk,\vk)$.
  \end{frame}

  \begin{frame}{Hardness for synthetic datasets~\cite{conf/stoc/DworkNRRV09}}{A
      view of an alleged synthetic dataset $X^m$}
    \fboxsep=0pt
    \noindent
      \begin{minipage}[l]{0.48\linewidth}

        \begin{center}
          $X^n=$
          \begin{tabular}{ | c | }
            \hline
            $(m_1,\sigma_1)$ \\ \hline
            $(m_2,\sigma_2)$ \\ \hline
            $(m_3,\sigma_3)$ \\ \hline
            $\vdots$ \\ \hline
            $(m_n,\sigma_n)$ \\ \hline
          \end{tabular}
          \end{center}
        
      \end{minipage}%
    \hfill%
      \begin{minipage}[r]{0.48\linewidth}
        \begin{center}
          $X^m=$
          \begin{tabular}{ | c | }
            \hline
            $(m_1,\sigma_1)$? \\ \hline
            $(m_2,\sigma_2)$? \\ \hline
            $(m_3,\sigma_3)$? \\ \hline
            $\vdots$ \\ \hline
            $(m_m,\sigma_m)$? \\ \hline
          \end{tabular}
          \end{center}
      \end{minipage}
      \vspace{1em}
      \begin{itemize}
        \item Counting query $q(X^n) = \frac{1}{n} \sum_{i=1}^n
          q_{\vk}(m_i,\sigma_i) = 1$
          \vspace{1em}
        \item Counting query $q(X^m) = \frac{1}{m} \sum_{i=1}^m
          q_{\vk}(m_i,\sigma_i) \approx 1$
        \vspace{1em}
        \item Utility $\land$ $(X^n \cap X^m) \neq \emptyset$\quad /OR/\quad
          forge sigs (efficiently) $\rightarrow$ break crypto++
        \end{itemize}
\end{frame}


\begin{frame}{Hard-to-sanitize distribution~\cite{conf/stoc/DworkNRRV09}}{A definition}
  Hard-to-sanitize distibutions on datasets:
  \begin{itemize}
  \item Sample a dataset $x$ 
  \item For all $q \in Q$,
  \item $\forall$ PPT sanitizers $A$ with output $y=A(x) \Longrightarrow$  PPT adversary $T$ s.t.:
    \begin{itemize}
    \item $\Pr[ |q(x) - q(y)| \leq \alpha \quad \land \quad T(y) \cap x = \emptyset] \leq \negl$
    \item $\Pr [ x_i \in T(y') ] \leq \negl$
          \\ $y' = A(x')$ and $x_i \not\in x$
    \end{itemize}
  \end{itemize}

  In words:
  \begin{itemize}
    \item being accurate without leaking elements has negl prob.
    \item extracting an element that is not in the dataset has negl prob.
  \end{itemize}
\end{frame}


\begin{frame}{Hard-to-sanitize
    distribution~\cite{conf/stoc/DworkNRRV09}}{Synthetic datasets. A TTS connection}
  \begin{itemize}
    \item (super-strong) signatures: $|Q| = \poly$, $|X| = \exp$
    \item PRFs: $|Q| = \exp$, $|X| = \poly$
    \item Both assume OWFs
    \end{itemize}

    \begin{itemize}
        \item What if $A(x)$ is an arbitrary output?
        \item Connection to traitor tracing schemes (TTS)~\cite{Chor94tracingtraitors}
    \end{itemize}
\end{frame}

\begin{frame}{Hardness results~\cite{conf/stoc/DworkNRRV09}}
  \begin{itemize}
    \item (super-strong) signatures: $|Q| = \poly$, $|X| = \exp$
    \item PRFs: $|Q| = \exp$, $|X| = \poly$
    \item Both assume OWFs
    \end{itemize}

    \begin{itemize}
        \item What if $A(x)$ is an arbitrary output?
        \item Connection to traitor tracing schemes (TTS) first
          defined~\cite{Chor94tracingtraitors}
    \end{itemize}

    \begin{itemize}
        \item TTS imply hard-to-sanitize distributions
        \item Hard-to-sanitize distributions imply TTS (one-shot + up to some parameters)
     \end{itemize}
\end{frame}

\begin{frame}{TTS implies hard-to-sanitize
    distributions~\cite{conf/stoc/DworkNRRV09}}{TTS definition}

  $t$-resilient (private-key) TTS
  \begin{itemize}
  \item Consists of $(\Gen, \Enc, \Dec, \Trace)$ 
  \item $(\Gen, \Enc, \Dec)$ is a semantically secure enc. scheme ($\sk_1,\ldots,\sk_n,\pk$)
  \item $\leq t$ users arbitraty combine their secret-keys $\longrightarrow$ decoder $D$
  \item $\Trace(D)$ with black-box access $\Longrightarrow$ trace back at least one user  
  \end{itemize}
  Can assume $t=n$.
\end{frame}

\begin{frame}{TTS implies hard-to-sanitize
    distributions~\cite{conf/stoc/DworkNRRV09}}{The reduction}
  \begin{itemize}
  \item Data universe X = Secret-Keys: $\zo^{\ksize(n,\secp)}$
  \item Query set     Q = Ciphertexts: $q_c(x_i) = \Dec(c,x_i)$
        \\(output LSB for counting queries)
  \item Both     $|Q|, |X| =  \exp(n, \secp)$
  \item ``Loose'' utility $\alpha < 1/2 \Rightarrow$ $\round{0 \pm \alpha} =0$
    and $\round{1 \pm \alpha} =1$
  \end{itemize}

  \begin{itemize}
  \item Utility $\Rightarrow$ Decoder $\Rightarrow$ Trace $\Rightarrow$ No-privacy
  % \item Trace $x'_i \not\in x \Rightarrow$ $n-1$ in $x'$ collude $$ Decoder = $x'_i$ and 
  \item Trace $\rightarrow x'_i \not\in x \Rightarrow$ Decoder = $x'_i$ $\Rightarrow$ Blame
    innocent user
  \end{itemize}
\end{frame}

\begin{frame}{TTS implies hard-to-sanitize
    distributions~\cite{conf/stoc/DworkNRRV09}}{Instatiate with~\cite{conf/eurocrypt/BonehSW06}}
  \begin{itemize}
  \item Secret-Key and Ciphertext $\Rightarrow$ size of X, Q
  \item Full collusion resilience~\cite{conf/eurocrypt/BonehSW06} 
  \item $\ksize=O(\secp)$\quad and \quad $\csize=O(\sqrt{n}\secp)$
  \item $|X| = 2^{\ksize}$ \quad and \quad $|Q| = 2^{\csize}$
  \end{itemize}
\end{frame}

\begin{frame}{Smaller TTS parameters~\cite{conf/tcc/KowalczykMUZ16}}{What is an
    iO scheme?}
\begin{definition}
A PPT algorithm iO is an indistinguishability obfuscator for a family of circuits
$\{C_{\secp}\}$ that satisfies the following properties:
\begin{itemize}
\item \textbf{Correctness:} For all $\secp$, $C \in \{C_{\secp}\}$ and $x$
  \[
       \Pr_{iO\,\,\textrm{coins}}[ iO(C)(x) = C(x) ] = 1
  \]
\item \textbf{Security:} For all $C_0,C_1 \in \{C_{\secp}\}$ such that for all
  $x$, $C_0(x) = C_1(x)$ and poly sized adversaries $\Adv$,
  \[
     | \Pr[ \Adv(iO(C_0(x)) = 1 ] - \Pr[ \Adv(iO(C_1(x)) = 1 ] | \leq \negl(\secp)
  \]
\end{itemize}
\end{definition}
We note that we are interested only in families of polynomial sized circuits.
\end{frame}

\begin{frame}{Smaller TTS parameters~\cite{conf/tcc/KowalczykMUZ16}}{What is an
    iO scheme?}
  \begin{itemize}
  \item iO($\cdot$) and iO(C(x)) are efficient (assume C(x) is efficient)
  \item \textbf{(correctness)} For all C: iO(C(x)) = C(x)
  \item \textbf{(security)} For all $C_0,C_1$ with $C_0(x)=C_1(x) \Longrightarrow iO(C_0(x)) \indc iO(C_1(x))$
  \end{itemize}
\end{frame}


\begin{frame}{Smaller TTS parameters~\cite{conf/tcc/KowalczykMUZ16}}{What is an
    puncturable PRF?}
  \begin{itemize}
  \item Allowed to evaluate on all but some ``punctured'' inputs
  \item ``Punctured'' inputs do not return PRF value
    \\Still return a pseudorandom value
  \item Can get them from GGM construction
  \end{itemize}
\end{frame}


\begin{frame}{Smaller TTS parameters~\cite{conf/tcc/KowalczykMUZ16}}{The old and the new}
\begin{itemize}
\item A PRG: $\zo^{\secp/2} \rightarrow \zo^{\secp}$
\item A puncturable PRF:~~~$\PRF_{\sk}:[n] \rightarrow
  \zo^{\secp}$
\item A twice puncturable PRF:~~~$\PRF_{\Enc}:[m] \rightarrow [n]$
\item An indistinguishability obfuscator iO.
\end{itemize}
\end{frame}


\begin{frame}{Smaller TTS parameters~\cite{conf/tcc/KowalczykMUZ16}}{Put
    everything together}
The scheme works in the following way.
\begin{itemize}
\item $\Setup(1^\secp)$.
  \begin{itemize}
  \item Sample contrained $\PRF_{\sk}$ and $\PRF_{\Enc}$
  \item $s_i = \PRF_{sk}(i)$. Let $O \sample iO(\prog)$.
  \item User's secret-key $\sk_i = (i, s_i, O)$ and the master key $\mk=\PRF_{\Enc}$.
\end{itemize}

\item $\Enc(j, \mk)$. Output $c \sample \PRF^{-1}_{\Enc}(j)$.
\item $\Dec(\sk_i, c)$. Output $O(c,i,s_i)$.
\end{itemize}

\begin{tcolorbox}[colframe=gray,colback=gray!30]
$\prog(c,i,s)$:

\hspace{1em} If $\PRG(s) \neq \PRG(\PRF_{\sk}(i))$, halt and output $\bot$.

\hspace{1em} Output $\mathbb{I}\{i \leq \PRF_{Enc}(c) \}$.
\end{tcolorbox}
\end{frame}


\begin{frame}{Smaller TTS parameters~\cite{conf/tcc/KowalczykMUZ16}}{The parameters}
\begin{itemize}
\item $|\sk_i| = \log n + \secp +|O| = \poly(\log n + \secp)$ (bits)
\item $c \in [m] \Rightarrow |c| = \log m \approx \log n^7 = \tO(\log n)$
\item \fbox{$|X| = \zo^{\poly(\log n + \secp)}$} \qquad \fbox{$|Q| = m\approx n^7 = \poly(\secp)$}
\item Can do vice-versa
\end{itemize}
\end{frame}

\begin{frame}{Open problems from~\cite{conf/tcc/KowalczykMUZ16} and~\cite{conf/tcc/BunZ16}}
\begin{itemize}
\item Replace iO with standard assumptions (i.e. LWE)
  \begin{itemize}
  \item Contrained \& Contraint Hiding PRFs (LWE~\cite{journals/iacr/BonehKM17,journals/iacr/CanettiC17})
  \end{itemize}
\item Reduce degree of $n^7$, i.e. either $|X|$ or $|Q|$. 
  \begin{itemize}
  \item VBB gets you $n^3$. 
  \item Hard as low as $n^{2+o(1)}$ for $|Q|$ or $|X|$? (Laplace, Histogram)
  \end{itemize}
\item Hardness results for PAC learning from crypto using ORE
  \begin{itemize}
      \item Threshold class. More ``natural'' classes?
      \item iO, mmaps(functional enc.), NIZKs
      \item Standard assumptions?
  \end{itemize}
\end{itemize}
\pause
\begin{center}
  {\Large Questions?}~\includegraphics[width=0.8cm]{trollface}
\end{center}
\end{frame}
